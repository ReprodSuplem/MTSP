\documentclass[conference]{IEEEtran}

\usepackage{algorithm}
\floatname{algorithm}{Modification}
\usepackage{algorithmic}
\usepackage{amssymb}
\usepackage{amsmath}

\begin{document}
	
% paper title
\title{\textsc{QMaxSAT} in MaxSAT Evaluation 2018}

% author names and affiliations
% use a multiple column layout for up to three different
% affiliations
\author{\IEEEauthorblockN{Aolong Zha}
\IEEEauthorblockA{\textit{Faculty of Information Science and Electrical Engineering} \\
  \textit{Kyushu University} \\
  744 Motooka, Nishi-ku, Fukuoka, Japan \\
  cyouryuuryuu@gmail.com}
}

% conference papers do not typically use \thanks and this command
% is locked out in conference mode. If really needed, such as for
% the acknowledgment of grants, issue a \IEEEoverridecommandlockouts
% after \documentclass

% for over three affiliations, or if they all won't fit within the width
% of the page, use this alternative format:
% 
%\author{\IEEEauthorblockN{Michael Shell\IEEEauthorrefmark{1},
%Homer Simpson\IEEEauthorrefmark{2},
%James Kirk\IEEEauthorrefmark{3}, 
%Montgomery Scott\IEEEauthorrefmark{3} and
%Eldon Tyrell\IEEEauthorrefmark{4}}
%\IEEEauthorblockA{\IEEEauthorrefmark{1}School of Electrical and Computer Engineering\\
%Georgia Institute of Technology,
%Atlanta, Georgia 30332--0250\\ Email: see http://www.michaelshell.org/contact.html}
%\IEEEauthorblockA{\IEEEauthorrefmark{2}Twentieth Century Fox, Springfield, USA\\
%Email: homer@thesimpsons.com}
%\IEEEauthorblockA{\IEEEauthorrefmark{3}Starfleet Academy, San Francisco, California 96678-2391\\
%Telephone: (800) 555--1212, Fax: (888) 555--1212}
%\IEEEauthorblockA{\IEEEauthorrefmark{4}Tyrell Inc., 123 Replicant Street, Los Angeles, California 90210--4321}}

% use for special paper notices
%\IEEEspecialpapernotice{(Invited Paper)}

\maketitle

\textsc{QMaxSAT} is a  satisfiability-based solver, which uses CNF encoding of pseudo-Boolean (PB) 
constraints \cite{DBLP:journals/jsat/KoshimuraZFH12}. 
The efficiency of MaxSAT solvers depends on critically on which SAT solver we use 
and how we encode the PB constraints. The \textsc{QMaxSAT} is obtained by adapting a CDCL based 
SAT solver \textsc{Glucose 3.0} \cite{DBLP:conf/ijcai/AudemardS09,10.1007/978-3-540-24605-3_37}. 
In addition, we introduce a new encoding method, called $n$-level modulo 
totalizer encoding in to our solver. This encoding is a hybrid between Modulo Totalizer (MTO) 
\cite{DBLP:conf/ictai/OgawaLHKF13} and Weighted Totalizer (WTO) \cite{hayata2015weighted}, 
incorporating the idea of mixed radix base \cite{DBLP:conf/tacas/CodishFFS11}. 

Let $\phi=
\{(C_{1},w_{1}),\dots,(C_{m},w_{m}),C_{m+1},\dots,C_{m+m'}\}$ be a
MaxSAT \cite{li2009maxsat} instance where $C_{i}$ is a soft clause with weight
$w_{i}$ $(i = 1,\dots,m)$ and $C_{m+j}$ is a hard
clause $(j = 1,\dots,m')$.
We added a new blocking variable, $b_{i}$, to each soft clause $C_{i} (i
= 1,\dots,m)$.
Solving the MaxSAT problem for $\phi$ is reduced to
finding a SAT model of
$\phi' = \{C_{1}\vee b_{1},\dots,C_{m}\vee b_{m},C_{m+1},\dots,C_{m+m'}\}$, 
which minimizes $\sum_{i=1}^{m} w_{i}b_{i}$.

Such SAT models are obtained using a SAT solver as follows:
Run the SAT solver to get an initial model and calculate
$k = \sum_{i} w_{i}b_{i}$ in it,
add PB constraint $\sum_{i} w_{i}b_{i} < k$,
and run the solver again. If $\phi'$ is unsatisfiable,
then $\phi$ is also unsatisfiable as the MaxSAT problem.
Otherwise, the process is repeated with the new smaller solution.
The latest model is a MaxSAT solution of $\phi$.
\textsc{QMaxSAT} leaves the manipulation of the PB constraints to \textsc{Glucose} by 
encoding them into SAT.

We introduce a hybrid encoding 
\cite{DBLP:conf/taai/ZhaAhyb} which inherits 
modular arithmetic from MTO and distinct combinations of weights from WTO. 
The latter is essentially the same as Generalized Totalizer, which 
only generate auxiliary variables for each unique combination of weights. 
We also enhanced the encoding by multi-level modulo 
arithmetic based on a mixed radix numeral system \cite{DBLP:conf/ictai/Zhamix13}. 
This encoding method always produces a polynomial-size CNF in the number of input variables. 

It is important to find a suitable mixed radix base with low time-consumption 
that reduces the number of auxiliary variables for our new encoding. 
We select the integer whose rate of divisibility is the highest for all weights\footnote{
Before selecting the next modulus, we update all the weights to their quotients of dividing 
the previous selected modulus. 
} as the suitable modulus for each digit. Furthermore, 
we also add other heuristics tailored in our implementation, such as evaluating and 
voting for the candidates of modulus, dynamically adjusting the lower limit of the 
required rate of divisibility, etc. 

\bibliographystyle{IEEEtran}
\bibliography{QMaxSAT}

\end{document}
